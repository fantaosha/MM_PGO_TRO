\noindent\textbf{Proof of \ref{prop::FsF1}.\;} We will prove $F(\Xk)=\sum_{\alpha\in\AA} F^{\ak}$ by induction.
\begin{enumerate}[leftmargin=0.45cm]
\item From \cref{eq::F}, it can be shown that
\begin{align}
&F(X^{(0)})\nonumber\\
=&\sum_{\alpha\in\AA}\sum_{(i,j)\in \aEE^{\alpha\alpha}}F_{ij}^{\aa}(X^{(0)})+
\sum_{\substack{\alpha,\beta\in\AA,\\\alpha\neq \beta}}\sum_{(i,j)\in \aEE^{\alpha\beta}} F_{ij}^{\ab}(X^{(0)})\nonumber\\
=&\sum_{\alpha\in\AA}\bigg(\sum_{(i,j)\in \aEE^{\alpha\alpha}}F_{ij}^{\aa}(X^{(0)}) +\nonumber\\
 &\quad\quad\quad\frac{1}{2}\sum_{\beta\in\NN_-^{\alpha}}\sum_{(i,j)\in \aEE^{\alpha\beta}} F_{ij}^{\ab}(X^{(0)})+\label{eq::FF0}\\
 &\quad\quad\quad\frac{1}{2}\sum_{\beta\in\NN_+^{\alpha}}\sum_{(i,j)\in \aEE^{\beta\alpha}} F_{ij}^{\ba}(X^{(0)})\bigg)\nonumber\\
=&\sum_{\alpha\in\AA} F^{\alpha(0)},\nonumber
\end{align}
where the last equality results from \cref{eq::Fa0}.
\vspace{0.25em}
\item Suppose $F(\Xk)=\sum_{\alpha\in\AA} F^{\ak}$ holds at $\sk$-th iteration. As a result of \cref{eq::Gakp}, we obtain
\begin{equation}\label{eq::GGa}
\begin{aligned}
&\sum_{\alpha\in\AA}G^{\akp}\\
=&\sum_{\alpha\in\AA} G^\alpha(\Xakp|\Xk) + \sum_{\alpha\in\AA}F^{\ak}\\
=&\sum_{\alpha\in\AA} G^\alpha(\Xakp|\Xk) + F(\Xk)\\
=& G(\Xkp|\Xk),
\end{aligned}
\end{equation}
where the second and third equality are due to $F(\Xk)=\sum_{\alpha\in\AA} F^{\ak}$ and \cref{eq::G2}, respectively. From \cref{eq::DGa} and
\begin{equation}
\nonumber
\|\Xkp-\Xk\|^2=\sum_{\alpha\in\AA}\|\Xakp-\Xak\|^2,
\end{equation}
it is straightforward to show
\begin{multline}\label{eq::DG}
\sum_{\alpha\in\AA} \Delta G^{\alpha}(\Xakp|\Xk)=\\
\sum_{\substack{\alpha,\beta\in\AA,\\\alpha\neq \beta}}\sum_{(i,j)\in \aEE^{\ab}} \left(F_{ij}^{\ab}(X)-E_{ij}^{\ab}(X|\Xk)\right)-\\
\frac{\xi}{2} \big\|\Xkp-\Xk\big\|^2.
\end{multline}
In addition, \cref{eq::Fakp,eq::GGa} suggest
\begin{equation}\label{eq::FGG}
\begin{aligned}
&\sum_{\alpha\in\AA}F^{\akp}\\
=&\sum_{\alpha\in\AA}G^{\akp}+\sum_{\alpha\in\AA} \Delta G^{\alpha}(\Xakp|\Xk)\\
=&G(\Xkp|\Xk) + \sum_{\alpha\in\AA} \Delta G^{\alpha}(\Xakp|\Xk).
\end{aligned}
\end{equation}
Substituting \cref{eq::G,eq::DG} into \cref{eq::FGG} yields.
\begin{multline}
\nonumber
\sum_{\alpha\in\AA}F^{\akp}= \sum_{\alpha\in\AA}\sum_{(i,j)\in \aEE^{\alpha\alpha}}F_{ij}^{\aa}(\Xkp)+\\
\sum_{\substack{\alpha,\beta\in\AA,\\\alpha\neq \beta}}\sum_{(i,j)\in \aEE^{\alpha\beta}} F_{ij}^{\ab}(\Xkp).
\end{multline}
We simplify the equation above with \cref{eq::F} and obtain
\begin{equation}
F(\Xkp)=\sum_{\alpha\in\AA}F^{\akp}.
\end{equation}
\item Therefore, it can be concluded that $F(\Xk)=\sum_{\alpha\in\AA} F^{\ak}$ holds for any $\sk\geq 0$.
\end{enumerate}
\vspace{0.5em}

\noindent\textbf{Proof of \ref{prop::FsF2}.\;} We will prove $\sF^{(\sk)} = \sum_{\alpha\in\AA}\sF^{\ak}$ by induction.
\begin{enumerate}[leftmargin=0.45cm]
\item Recall from \cref{eq::sFa0,eq::FF0,eq::lFk} that $\sF^{(0)}=F(X^{(0)})$, $\sF^{\alpha(0)}=F^{\alpha(0)}$ and $F(X^{(0)})=\sum_{\alpha\in\AA} F^{\alpha(0)}$, which immediately yields
\begin{equation}
\sF^{(0)}=F(X^{(0)}).
\end{equation}
\item Suppose $\sF^{(\sk)}=\sum_{\alpha\in\AA} \sF^{\ak}$ holds at $\sk$-th iteration. As a result of \cref{eq::lFk}, we obtain
\begin{equation}
\nonumber
\sF^{(\skp)} = (1-\eta)\cdot\sF^{(\sk)}+\eta\cdot F(\Xkp).
\end{equation}
Note that \cref{prop::FsF}\ref{prop::FsF1} suggests $F(\Xkp)=\sum_{\alpha\in\AA}F^{\akp}$. Apply $\sF^{(\sk)}=\sum_{\alpha\in\AA} \sF^{\ak}$ and $F(\Xkp)=\sum_{\alpha\in\AA}F^{\akp}$ on the right-hand side of the equation above results in
\begin{multline}
 \sF^{(\skp)}=(1-\eta)\cdot\sum_{\alpha\in\AA}\sF^{\ak} +\\
  \eta\cdot\sum_{\alpha\in\AA}F^{\akp}=\sum_{\alpha\in\AA}\sF^{\akp},
\end{multline}
where the last equality is due to \cref{eq::lFakp}.
\item Therefore, it can be concluded that $\sF^{(\sk)}=\sum_{\alpha\in\AA} \sF^{\ak}$ holds for any $\sk\geq 0$.  The proof is completed.
\end{enumerate}

\vspace{0.5em}

\noindent\textbf{Proof of \ref{prop::FsF4}.\;} \cref{prop::upper} indicates $E_{ij}^{\ab}(X|\Xkm)\geq F_{ij}^{\ab}(X)$ and $E_{ij}^{\ba}(X|\Xkm)\geq F_{ij}^{\ba}(X)$, from which and \cref{eq::DGa} we obtain 
\begin{equation}\label{eq::DGal}
	\Delta G^\alpha(\Xa|\Xkm)\leq 0
\end{equation}
as long as $\xi\geq 0$. From \cref{eq::DGal,eq::Fakp}, it is immediate to conclude 
\begin{equation}
F^{\akp}\leq G^{\akp}
\end{equation}
for any $\sk\geq 0$. If $G^{\akp}\leq \sF^{\ak}$, the equation above further suggests
\begin{equation}\label{eq::FGlF}
F^{\akp}\leq G^{\akp}\leq \sF^{\ak}.
\end{equation}
From \cref{eq::lFakp}, we obtain
\begin{equation}\label{eq::sFakp}
\sF^{\akp} = (1-\eta)\cdot \sF^{\ak} + \eta\cdot F^{\akp},
\end{equation}
where note that $\eta\in(0,1]$. Thus, we conclude that $\sF^{\akp}$ is a convex combination of $\sF^{\ak}$ and $F^{\akp}$, which and \cref{eq::FGlF} lead to
\begin{equation}
F^{\akp}\leq \sF^{\akp}\leq \sF^{\ak}.
\end{equation}
The proof is completed.

